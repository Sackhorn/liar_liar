\documentclass{article}
\usepackage[polish]{babel}
\usepackage[utf8]{inputenc}
\usepackage[T1]{fontenc}
\usepackage[a4paper, total={6in, 10in}]{geometry}
\usepackage[T1]{fontenc}
\usepackage{cite}
\usepackage{graphicx}
\usepackage{caption}
\usepackage{listing}
\usepackage{subcaption}
\usepackage{amsmath}
\usepackage{algorithm}% http://ctan.org/pkg/algorithms
\usepackage{algcompatible}% http://ctan.org/pkg/algorithmicx
\usepackage[noend]{algpseudocode}


\graphicspath{ {images/} }

\immediate\write18{bibtex \jobname}

\newcommand\todo[1]{\textcolor{red}{#1}}


\title{
Praca Inżynierska
\begin{large}
\\Temat: Aplikacja do testowania odporności
modeli klasyfikacyjnych na ataki z użyciem złośliwych danych
\end{large}}

\date{25 Sierpnia 2019}
\author{Jan Ambroziak \\ Opiekun Pracy: dr.inż Paweł Zawistowski}

\begin{document}
\maketitle

\

\section{Wstep}
Wraz z rozwojem i upowszechnianiem się modeli klasyfikujących obrazy graficzne,
pojawiają się nowe zagrożenia związane z bezpieczeństwem tej klasy rozwiązań.
Jednym z nich są złośliwe dane, które powodować mogą,
że pozornie dobrze działający klasyfikator zaczyna udzielać błędnych odpowiedzi.
Może to stanowić realne zagrożenie nawet dla życia i zdrowia ludzkiego
np. w przypadku gdy taki atak dotyczy systemu automatycznie sterującego pojazdem.

\section{Cel}
\label{sec:target}
Celem pracy inżynierskiej jest stworzenie aplikacji która ma wspierać testowanie
odporności modeli klasyfikacyjnych na ataki z użyciem złośliwych danych
Dla dostarczonego w narzuconej postaci modelu klasyfikatora możliwe będzie
przeprowadzenie ataku przy pomocy jednej z dostępnych metod.

\section{Użyte narzędzia}
Od czasu opublikowania przez firmę Google biblioteka Tensorflow \todo{citation needed}
stała się jednym z wiodących narzędzi wykorzystywanych w branży. Prostota API i intuicyjność
języka Python, akceleracja obliczeń z wykorzystaniem procesorów graficznych oraz dobra integracja z
popularnymi bibliotekami dostępnymi dla języka Python czyni Tensorflow idealnym kandydatem.
W trakcie pracy wykorzystywałem głównie wysokopoziomowy interfejs Keras, który pozwala na szybkie
prototypowanie oraz na małą złożoność kodu źródłowego.
Platforma testowa jakiej użyto to:
\begin{itemize}
    \item CPU: Intel i5-2500K
    \item GPU: Nvidia GeForce 1060 GTX 6GB
    \item RAM: 8GB DDR3
    \item System Operacyjny: Windows 10
    \item Tensorflow wersja 2.0.0b1
    \item Interpreter Python 3.6.8 wersja 64bit
\end{itemize}
Wszystkie pozostałe biblioteki Python, wraz z wersjami, wykorzystane w projekcie, znajdują się w pliku
requirements.txt w folderze projektu.

\section{Modele i dane testowe}
Aby sprawdzić poprawność implementacji oraz skuteczność ataków stworzono
kilka modeli klasyfikacyjnych o różnej złożoności, opartych o popularne zbiory danych
wykorzystywanych jako platforma testowa w publikacjach pokrewnych tematyką.

    \subsection{MNIST}
    Zbiór danych MNIST \todo{citation needed} to najprawdopodobniej najpopularniejszy zbiór związany z
    tematyką nauczania maszynowego.
    Zawiera on 60,000 obrazów o rozmiarach 28 na 28 pikseli, w skali szarości, przedstawiających
    odręcznie narysowane cyfry. Zbiór dzieli się na 50,000 przykładów używanych do
    trenowania modelu oraz 10.000 służących do testowania.

    \subsection{CIFAR10}
    Zbiór CIFAR10 \todo{citation needed} zawiera kolorowe obrazy o rozdzielczości 32 na 32 piksele,
    podzielone na 10 klas takich jak np. koń, samolot czy żaba. Podobnie jak zbiór MNIST, CIFAR10 zawiera
    60,000 obrazów podzielonych na 50,000 przykładów używanych do treningu jak i 10,000 do testowania.
    Aby usunąć problem nadmiernego dopasowania modeli obrazy z zbioru CIFAR10 używane
    do trenowania są losowo modyfikowane. Każdy obraz ma szanse 0.25 \todo{tutaj uzupełnić} na
    modyfikacje natężenia, nasycenia, kontrastu oraz odcienia. Dodatkowo każdy obraz może zostać obrócony bądź odbity.

    \subsection{ImageNet}
    \todo{Tutaj uzupełnić}




\section{Rodzaje Ataków}
Ideą ataków adwersaryjnych \todo{tu dodać jakąś definicje} jest osiągnięcie jednego z wymienionych celów:
\begin{enumerate}
    \item kaczka
    \item Zmniejszenie prawdopodobieństwa z jakim model klasyfikuje przykład
    \item Zła klasyfikacja przykładu wejściowego (niezaleznie od końcowej klasy)
    \item Przyporządkowanie przykładu wejściowego do złej, zadanej z góry klasy
    \item Tutaj było coś jeszcze \todo{Sprawdzić i dodać cytowanie}
\end{enumerate}

Możemy też rozróżnić ataki z uwagi na to jakie dane są dostępne dla algorytmu atakującego:
\begin{enumerate}
    \item Dostępny jest gradient, funkcja kosztu i funkcja trenująca
    \item Dostepna jest tylko funkcja kosztu
    \item Dostępny jest przykład i coś tam
    \item Dostępne są tylko dane wejściowe i klasa do jakiej model przyporządkowuje dany przykład
    \item \todo{poprawić powyższe i dodać cytowanie}
\end{enumerate}


\section{Wykorzystane Ataki}
W projekcie zaimplementowanych zostało kilka ataków które zostały opisane w anglojęzycznych publikacjach.
Poniżej znajdują się ich opisy, uwagi dotyczące implementacji, przykłady zastosowania oraz wyniki testów
zastosowanych do oceny ataków. Szersza dyskusja dotycząca użyteczności oraz porównanie znajdują się w sekcji \ref{comparison}

\subsection{FGSM}
    FGSM czyli Metoda szybkiego znaku gradientu (z ang. Fast Gradient Sign Method) opsiana w artykule
    pod tytułem \textit{Explaining and Harnessing Adversarial Examples}\cite{harnessing} to jedna z pierwszych
    opisanych metod ataku na modele klasyfikacyjne wykorzystujący złośliwe dane.
    Aby wygenerować złośliwy przykład metoda ta wymaga znajomości gradientu funkcji kosztu obliczanego względem danych
    wejściowych oznaczonego jako

    \begin{equation}
    \(\nabla_{x} J(x, y)\).
    \qquad\text{gdzie}
    \begin{tabular}[t]{ll}
    $x$   & -- dane wejściowe \\
    $y_{prawdziwe}$   & -- klasa do której należą dane wejściowe\\
    $J(x, y_{prawdziwe})$  & -- funkcja kosztu \\
    $\nabla_{x}$  & -- gradient względem danych wejściowych \\
    \end{tabular}
    \end{equation}

    Idea metody polega na dodaniu bądź odjęciu małej, arbitralnie ustalonej, wartości (oznaczanej przez \(\epsilon\)) do
    danych wejściowych w zależności od znaku gradientu tak aby zwiększyć wartość funkcji kosztu,
    doprowadzając do niepoprawnej klasyfikacji danych wejściowych zmodyfikowanych w niewielkim stopniu.
    Możemy zapisać to działanie w następujący sposób.

    \begin{equation}
    x' = x + \epsilon\operatorname{sign}(\nabla_{x} J(x, y_{prawdziwe}))
    \qquad\text{gdzie}
    \begin{tabular}[t]{ll}
    $x'$  & -- spreparowany przykład ze złośliwymi danymi \\
    \end{tabular}
    \end{equation}

    Analogiczny przykład do tego znajdującego się w oryginalnej publikacji\cite{harnessing} znajduje się poniżej.
    \todo{tutaj wstawić przykład z dodawaniem obrazków}

    Podstawową wadą opisywanej metody jest fakt że nie mamy wpływu na to jaka będzie klasa wyjściowa
    spreparowanych przez nas danych wejściowych. Kolejną jest to że niektóre dane i modele wymagają od nas
    doboru wysokiej wartości parametru $\epsilon$ aby być w stanie spreparować pożądane dane, co z kolei powoduje dużą
    różnicę między danymi wejściowymi a tymi utworzonymi w toku stosowania metody.
    W publikacji pod tytułem \textit{Adversarial examples in the physical world}\cite{DBLP:journals/corr/KurakinGB16}
    autorzy opisują kilka metod pochodnych do FGSM.

    \subsubsection{I-FGSM}
    I-FGSM czyli Iteracyjna Metoda Szybkiego Znaku Gradientu (z ang. Iterative Fast Gradient Sign Method) to metoda
    które rozwiązuje problem konieczności doboru wysokiej wartości $\epsilon$ poprzez zastosowanie metody FGSM kilkakrotnie
    ,jednocześnie zapewniając pod koniec każdego kroku że przyogotowany każdy piksel obrazu nie różni się od oryginału o więcej
    niż \epsilon, z niższą wartością parametru $\epsilon$ niż wymagane oryginalnie w metodzie jednokrokowej.
    Pozwala to  potencjalnie ogarniczyć różnicę między preparowanymi przez nas danymi a oryginalnymi danymi wejściowymi
    oraz przerwanie działania metody kiedy model przestania poprawnie klasyfikować dane wejściowe.

    \begin{algorithm}
    \caption{I-FGSM}\label{IFGSM}
    \begin{algorithmic}[1]
    \State $i \gets 0$
    \While{$i <  i_{max}$}
        \State $x' = x + \alpha\operatorname{sign}(\nabla_{x} J(x, y_{prawdziwe}))$
        \State $i \gets i+1$
        \State $Przytnij(x', \epsilon)$
    \EndWhile
    \end{algorithmic}
    \end{algorithm}

    \subsubsection{LL-FGSM}
    LL-FGSM (z ang. Least Likely FGSM) to metoda pozwalająca nam na spreparowanie danych które będę przydzielane do
    konkretnej klasy (ozn. $y_{celu}$), a nie tylko na niepoprawną klasyfikacje. W tym wypadku zamiast starać się
    zmieniać wartości pikseli obrazu zgodnie z kierunkiem gradientu funkcji kosztu dla prawidłowej klasy
    staramy się zmieniać wartości pikseli przeciwnie do kierunku gradientu funkcji kosztu dla klasy $y_{celu}$.
    Intuicyjnie można to opisać jako nie oddalanie się od prawdziwej klasy obrazu, a jako przybliżanie się do zadanej
    przez nas klasy.
    Autorzy metody zwracają uwagę na przydatność tej metody
    w przypadku gdy korzystamy z modeli operujących wieloma klasami i gdzie różnice między obiektami z różnych klas mogą
    być bardzo niewielkie (np. między rasami psów). Metoda ta jest także iteracyjna i zapewnie różnice między
    odpowiednimi pikselami obrazów nie większą niż $\epsilon$ więc można uznać ją za rozszerzenie metody I-FGSM.

    \begin{algorithm}
    \caption{LL-FGSM}\label{LLFGSM}
    \begin{algorithmic}[1]
    \State $i \gets 0$
    \While{$i <  i_{max}$}
        \State $x' = x - \alpha\operatorname{sign}(\nabla_{x} J(x, y_{celu}))$
        \State $i \gets i+1$
        \State $Przytnij(x', \epsilon)$
    \EndWhile
    \end{algorithmic}
    \end{algorithm}

\subsubsection{Implementacja}

\subsection{DeepFool}
W publikacji \textit{DeepFool: a simple and accurate method to fool deep neural networks}\cite{DBLP:journals/corr/Moosavi-Dezfooli15}
autorzy proponują metodę alternatywną do FGSM mającą minimalizować wprowadzaną do danych wejściowych perturbacje jednocześnie
osiągając zmianę klasy do której model przyporządkowuje dane wejściowe. Punktem wyjściowym rozważań autorów było założenie
że model z którego korzystamy jest przekształceniem afinicznym, opracowanie metody pozwalającej na
uzyskanie zminimalizowanej perturbacji, a następnie zgeneralizowanie metody dla modeli nie będących przekształceniami afinicznymi.
Wadą DeepFool jest brak możliwości narzucenia klasy do której chcielibyśmy aby przyporządkowywane były nasze zmodyfikowane dane.
Metoda ta jest nieco bardziej złożona obliczeniowo jako że pojedynczy krok algorytmu wymaga od nas obliczenia gradientu dla
prawdopodobieństwa każdej klasy względem danych wejściowych, co przy zbiorach danych z wieloma klasami
takich jak np. ImageNet może być problematyczne. Poniżej znajduje się pseudokod opisujący metodę DeepFool dla modelu
wieloklasowego.

\subsubsection{Implementacja}


\subsection{L-BFGS}
Metoda opisana w \textit{Intriguing properties of neural networks}\cite{DBLP:journals/corr/SzegedyZSBEGF13}
to próba rozwiązania poniższego problemu optymalizacyjnego w celu uzyskania minimalnej perturbacji danych wejściowych
która powoduje niepoprawną klasyfikacje danych przez model:
    \begin{equation}
    \min{\| r\|_{2}}
    \qquad\text{przy warunkach:}
    \begin{tabular}[t]{ll}
    f(x + r) = y_{celu} \\
    x+r \in [0,1]^{m} \\
    \end{tabular}
    \qquad\text{gdzie:}
    \begin{tabular}[t]{ll}
    r - perturbacja \\
    \end{tabular}
    \end{equation}
Znalezienie dokładnego rozwiązania powyższego problemu jest skomplikowane, dlatego autorzy postanowili szukać aproksymacji
rozwiązania poprzez liniowe przeszukiwanie w celu znalezienia najmniejszej wartości parametru $c > 0$ dla którego spełniony
zostaje warunek $f(x+r) = y_{celu}$ gdzie $r$ uzyskujemy z zastosowania algorytmu L-BFGS dla poniższego problemu:
    \begin{equation}
    \min{c| r| + J(x+r, y_{celu})}
    \qquad\text{przy warunkach:}
    \begin{tabular}[t]{ll}
    x+r \in [0,1]^{m} \\
    \end{tabular}
    \end{equation}

\subsubsection{Implementacja}


\subsection{JSMA}
Idea ataku JSMA (Jacobian Saliency Map Attack) opisanego w
\textit{The Limitations of Deep Learning in Adversarial Settings}\cite{DBLP:journals/corr/PapernotMJFCS15}
polega na utworzeniu mapy istotności (z ang. Saliency Map) pikseli obrazu dla każdej z klas.
Dzięki temu jesteśmy w stanie określić jak dane piksele w obrazie wpływają na określanie prawdopodobieństwa należenia
do danej klasy przez model. Sposób tworzenia mapy istotności jest zamieszczony poniżej:
\begin{equation}\label{jsma+}
S^{+} ( \mathbf { x } , y ) [ i ] = \left\{ \begin{array} { c } { 0 \text { jeśli } \frac { \partial \mathbf {f} _ { y } ( \mathbf {x} ) } { \partial \mathbf {x} _ { i } } < 0 \text { lub } \sum _ { j \neq y } \frac { \partial \mathbf {f} _ { j } ( \mathbf {x} ) } { \partial \mathbf {x} _ { i } } > 0 } \\ { \left( \frac { \partial \mathbf {f} _ { y } ( \mathbf {x} ) } { \partial \mathbf {x} _ { i } } \right) \left| \sum _ { j \neq y } \frac { \partial \mathbf {f} _ { j } ( \mathbf {x} ) } { \partial \mathbf {x} _ { i } } \right| \text { w.p.p } } \end{array} \right.
\qquad\text{gdzie:}
\begin{tabular}[t]{ll}
x - dane wejściowe \\
${f} _ { y } ({x})$ - prawd. przynależności x do klasy y \\
$x_{i}$ - i-ty piksel obrazu wejściowego x \\
S({x},y)[i] - wartość i-tego piksel dla klasy y
\end{tabular}
\end{equation}

W powyżej zdefiniowanej mapie nieujemne wartości oznaczają poziom wpływu zwiększenia intensywności danego piksela $i$
na przynależność dla danej klasy $y$.
Warunek $\frac { \partial \mathbf {f} _ { y } ( \mathbf {x} ) } { \partial \mathbf {x} _ { i } } < 0$
zapewnia że nie będziemy rozpatrywać pikseli które mają negatywny wpływ na wartość prawdopodobieństwa należenia przykładu
do klasy $y$.
Natomiast warunek $\sum _ { j \neq y } \frac { \partial \mathbf {f} _ { j } ( \mathbf {x} ) } { \partial \mathbf {x} _ { i } } > 0 $
zapewnia że nie będziemy rozpatrywać pikseli które mają pozytywny wpływ na wartość prawdopodobieństwa należenia przykłady do klas
innych niż narzucona przez nas klasa $y$. Autorzy metody proponują też analogiczną mapę istotności w która
zamiast określać wpływ zwiększenia intensywności pikseli na przynależność przykładu do danej klasy określa wpływ zmniejszania
intensywności piksela na przynależność do danej klasy.

\begin{equation}\label{jsma-}
S^{-} ( \mathbf { x } , y ) [ i ] = \left\{ \begin{array} { c } { 0 \text { jeśli } \frac { \partial \mathbf {f} _ { y } ( \mathbf {x} ) } { \partial \mathbf {x} _ { i } } > 0 \text { lub } \sum _ { j \neq y } \frac { \partial \mathbf {f} _ { j } ( \mathbf {x} ) } { \partial \mathbf {x} _ { i } } < 0 } \\ { \left|( \frac { \partial \mathbf {f} _ { y } ( \mathbf {x} ) } { \partial \mathbf {x} _ { i } } )\right  \left| \sum _ { j \neq y } \frac { \partial \mathbf {f} _ { j } ( \mathbf {x} ) } { \partial \mathbf {x} _ { i } } \right \text { w.p.p } } \end{array} \right.
\end{equation}

W praktyce jednak większość pikseli nie spełnia warunków określonych w pierwszej linii podanych równań,
dlatego autorzy zastosowali metodę wyboru par pikseli $p_1$ i $p_2$ zamiast pojedynczego piksela.
W tym celu stosowana jest opisana poniżej metoda.

\begin{equation} \label{saliency_map}
\arg \max _ { \left( p _ { 1 } , p _ { 2 } \right) } \left( \sum _ { i = p _ { 1 } , p _ { 2 } } \frac { \partial \mathbf { f } _ { y } ( \mathbf { x } ) } { \partial \mathbf { x } _ { i } } \right) \times \left| \sum _ { i = p _ { 1 } , p _ { 2 } } \sum _ { j \neq y } \frac { \partial \mathbf { f } _ { j } ( \mathbf { x } ) } { \partial \mathbf { x } _ { i } } \right|
\end{equation}

O ile rozwiązuje to problem znalezienia pikseli które spełniają warunki o tyle metoda ta ma większą złożoność obliczeniową
z uwagi na to że musimy rozpatrzyć wszystkie możliwe pary pikseli zamiast tylko pojedynczych pikseli.
Opisany poniżej algorytm oddaje istotę opisanej przez autorów metody.

\begin{algorithm}
\caption{JSMA}\label{JSMA}
\begin{algorithmic}[1]
\State $x' \gets x$
\While{$f(x') \neq  y_{celu}\ \& \ i < i_{max}$}
    \State $p_1, p_2$ = S$(x',y_{celu})$ \Comment przez S rozumiemy (\ref{saliency_map})
    \State zmodyfikuj $p_1$ i $p_2$ o $\theta$
    \State jeśli $p_1$ lub $p_2$ wynosi 0 albo 1 usuń je z listy pikseli
    \State $i \gets i+1$
\EndWhile
\State \textbf{return} $x'$
\end{algorithmic}
\end{algorithm}

Istnieje wiele różnych wariantów metody JSMA, których zasadnicze działanie nie różni się bardzo od tego opisanego powyżej.
\textit{Maximal Jacobian-based Saliency Map Attack}\cite{DBLP:journals/corr/abs-1808-07945} przeprowadza bardzo
zwięzłe podsumowanie różnych wariantów metody JSMA. Warto tutaj przytoczyć kilka różnic między opisanymi w tej publikacji
wariantami JSMA.

\subsubsection{JSMA+ i JSMA-}
Autorzy \textit{Maximal Jacobian-based Saliency Map Attack}\cite{DBLP:journals/corr/abs-1808-07945} wprowadzają rozróżnienie
pomiędzy JSMA+ a JSMA- w zależności od tego czy intensywność pikseli jest zmniejszana i wykorzystywane są warunki~(\ref{jsma-})
czy też intensywność pikseli jest zwiększana i wykorzystywane są warunki~(\ref{jsma+})

\subsubsection{JSMA-F i JSMA-Z}
W publikacji \textit{Towards Evaluating the Robustness of Neural Networks}\cite{DBLP:journals/corr/CarliniW16a}
autorzy proponują rozróżnienie pomiędzy JSMA-F, które w metodzie~(\ref{saliency_map}) wykorzystuje do obliczania pochodnej cząstkowej
wyjście funkcji softmax stosowanej zazwyczaj jako ostatnia warstwa modelu, a JSMA-Z które zamiast tego wykorzystuje wektor
wyjściowy przedostatniej warstwy określany często jako zazwyczaj jako logits.

\subsubsection{NT-JSMA}
NT-JSMA czyli wersja JSMA bez zadanej klasy którą chcemy osiągnąć (z ang. Non Targeted JSMA). To postulowany przez autorów
\textit{Maximal Jacobian-based Saliency Map Attack}\cite{DBLP:journals/corr/abs-1808-07945} wariant metody JSMA
który poprzez zdjęcie ograniczenia polegającego na tym że spreparowana dane mają być klasyfikowane jako należące do zadanej
klasy ma umożliwić zmniejszenie perturbacji wymaganej do zmiany klasyfikacji danych przez model. To usprawnienie niesie
ze sobą jednak dodatkowy koszt obliczeniowy jako że w każdej iteracji rozpatrujemy $S^{-}()$ bądź $S^{+}()$ nie dla
jednej zadanej klasy tylko dla wszystkich.

\subsubsection{M-JSMA}
M-JSMA\cite{DBLP:journals/corr/abs-1808-07945} to wariant metody JSMA który łączy w sobie metodę


\subsubsection{Implementacja}



\subsection{Carlini & Wagner}
First in \cite{DBLP:journals/corr/CarliniW16a}
\subsubsection{Implementacja}

\subsection{Evolutionary Attack}
First in \cite{DBLP:journals/corr/NguyenYC14}
\subsubsection{Implementacja}


\section{Porównanie}\label{comparison}







%\section{Praca w ciągu semestru}
%\label{sec:work}
%
%Główną technologią na której opiera się moja praca jest biblioteka tensorflow\cite{tensorflow}.
%Główną zaletą jest możliwość akceleracji obliczeń z wykorzystaniem biblioteki CUDA i cuDNN, które
%umożliwiają zrównoleglenie obliczeń na karcie graficznej.
%Ponieważ jednym z założeń pracy inżynierskiej jest znajomość modelu, to rodzaj
%ataków jakimi będę się zajmuję to tak zwane ataki "white-box".
%Jedna z najbardziej znanych metod tworzenia złośliwych danych w to
%Fast Gradient Sign Method (w skrócie FGSM) opisana w \cite{harnessing}.
%Jest to też metoda którą udało mi się zaimplementować na wytrenowanych przeze mnie
%modelach klasyfikacyjnych (patrz punkt ~\ref{sec:fgsm}).
%
%Innymi metody nad których implementacją pracuje są
%Jacobian-based Saliency Map Attack (JSMA)\cite{DBLP:journals/corr/PapernotMJFCS15}
%oraz DeepFool\cite{DBLP:journals/corr/Moosavi-Dezfooli15}.
%Po zakończeniu implementacji i wykazaniu ich skutecznośći chciałbym
%przetestować ich działanie dla bardziej skomplikowanych modeli klasyfikacyjnych
%niż te z których korzystałem dotychczas(patrz punkt ~\ref{sec:fgsm}).
%Końcowym zadaniem będzie utworzenie prostego interfejsu do obsługi aplikacji
%(linia komend, bądź interfejs graficzny) oraz przetestowanie ataków na już
%wytrenowanych modelach dostępnych publicznie


%\section{Ataki z użyciem Fast Gradient Sign Method}
%\label{sec:fgsm}
%W celu sprawdzenia poprawności działania mojej implementacji metody FGSM utworzyłem
%i wytrenowałem dwa proste modele klasyfikacyjne będące
%splotowymi sieciami neuronowymi dla zbiorów CIFAR-10\cite{cifar_10} oraz MNIST\cite{mnist}
%mające praktycznie taką samą strukturę oprócz wymiarów.
%Struktura operacji ewaluacji uzyskana za pomocą narzędzia tensorboard zarówno dla
%zbioru MNIST i CIFAR-10 znajduje się poniżej
%
%\newpage
%\vfill
%
%\begin{figure}[ht]
%  \begin{subfigure}{.55\textwidth}
%    \includegraphics[width=\textwidth]{mnist_structure}
%    \caption{MNIST}
%    \centering
%  \end{subfigure}
%  \begin{subfigure}{.55\textwidth}
%    \includegraphics[width=\textwidth]{cifar_structure}
%    \caption{CIFAR}
%    \centering
%  \end{subfigure}
%\end{figure}
%
%\vfill
%\clearpage
%
%Poniżej można zobaczyc porównanie prawdopodobieństw przydziału do klasy dla
%orignalnych danych wejściowych oraz dla spreparownych złośliwych danych.
%Dla zbioru MNIST znacznie większy wpływ na skuteczność metody ma poziom
%perturbacji niż liczba dokonanych iteracji. Zadowlajace wyniki pojawiają
%się dopiero dla perturbacji na poziomie ~40 procent.
%W przypadku zbioru CIFAR-10 można aby uzyskać przypisanie spreparowanych danych
%do zadanej przez nas klasy z wysokim prawdopodobieństwem wystarczy 200 iteracji
%przy poziomie perturbacji ~10 procent.
%
%% \newpage
%% \vfill
%\begin{figure}[ht]
%  \begin{subfigure}{.55\textwidth}
%    \includegraphics[width=\textwidth]{mnist_orig}
%    \caption{MNIST Klasyfikacja oryginalnego obrazu}
%    \centering
%  \end{subfigure}
%  \begin{subfigure}{.55\textwidth}
%    \includegraphics[width=\textwidth]{mnist_adver}
%    \caption{MNIST Klasyfikacja spreparowanego obrazu dla zadanej klasy 1, 200 iteracji}
%    \centering
%  \end{subfigure}
%  \begin{subfigure}{.55\textwidth}
%    \includegraphics[width=\textwidth]{mnist_pert}
%    \caption{MNIST wprowadzona perturbacja}
%    \centering
%  \end{subfigure}
%\end{figure}
%\begin{figure}[ht]
%  \begin{subfigure}{.55\textwidth}
%    \includegraphics[width=\textwidth]{cifar_orig.png}
%    \caption{CIFAR Klasyfikacja oryginalnego obrazu}
%    \centering
%  \end{subfigure}
%  \begin{subfigure}{.55\textwidth}
%    \includegraphics[width=\textwidth]{cifar_adver}
%    \caption{CIFAR Klasyfikacja spreparowanego obrazu dla zadanej klasy 1, 200 iteracji}
%    \centering
%  \end{subfigure}
%  \begin{subfigure}{.55\textwidth}
%    \includegraphics[width=\textwidth]{cifar_pert}
%    \caption{CIFAR wprowadzona perturbacja}
%    \centering
%  \end{subfigure}
%\end{figure}
%
%% \vfill
%\clearpage





\bibliography{pdi}
\bibliographystyle{ieeetr}

\end{document}
