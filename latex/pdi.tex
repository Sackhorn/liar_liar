\documentclass{article}
\usepackage[polish]{babel}
\usepackage[utf8]{inputenc}
\usepackage[T1]{fontenc}
\usepackage[a4paper, total={6in, 10in}]{geometry}
\usepackage[T1]{fontenc}
\usepackage{cite}
\usepackage{graphicx}
\usepackage{caption}
\usepackage{listing}
\usepackage{subcaption}

\graphicspath{ {images/} }

\immediate\write18{bibtex \jobname}


\title{
Sprawozdanie z pracy w ramach Pracowni Dyplomowej Inżynierskiej 2
\begin{large}
\\Temat: Aplikacja do testowania odporności
modeli klasyfikacyjnych na ataki z użyciem złośliwych danych
\end{large}}

\date{21 Czerwca 2018}
\author{Jan Ambroziak \\ Opiekun Pracy: dr.inż Paweł Zawistowski}

\begin{document}
\maketitle

\section{Wstep}
Wraz z rozwojem i upowszechnianiem się modeli klasyfikujących obrazy graficzne,
pojawiają się nowe zagrożenia związane z bezpieczeństwem tej klasy rozwiązań.
Jednym z nich są złośliwe dane, które powodować mogą,
że pozornie dobrze działający klasyfikator zaczyna udzielać błędnych odpowiedzi.
Może to stanowić realne zagrożenie nawet dla życia i zdrowia ludzkiego
np. w przypadku gdy taki atak dotyczy systemu automatycznie sterującego pojazdem.

\section{Cel}
\label{sec:target}
Celem pracy inżynierskiej jest stworzenie aplikacji która ma wspierać testowanie
odporności modeli klasyfikacyjnych na ataki z użyciem złośliwych danych
Dla dostarczonego w narzuconej postaci modelu klasyfikatora możliwe będzie
przeprowadzenie ataku przy pomocy jednej z dostępnych metod.

\section{Praca w ciągu semestru}
\label{sec:work}

Główną technologią na której opiera się moja praca jest biblioteka tensorflow\cite{tensorflow}.
Główną zaletą jest możliwość akceleracji obliczeń z wykorzystaniem biblioteki CUDA i cuDNN, które
umożliwiają zrównoleglenie obliczeń na karcie graficznej.
Ponieważ jednym z założeń pracy inżynierskiej jest znajomość modelu, to rodzaj
ataków jakimi będę się zajmuję to tak zwane ataki "white-box".
Jedna z najbardziej znanych metod tworzenia złośliwych danych w to
Fast Gradient Sign Method (w skrócie FGSM) opisana w \cite{harnessing}.
Jest to też metoda którą udało mi się zaimplementować na wytrenowanych przeze mnie
modelach klasyfikacyjnych (patrz punkt ~\ref{sec:fgsm}).

Innymi metody nad których implementacją pracuje są
Jacobian-based Saliency Map Attack (JSMA)\cite{DBLP:journals/corr/PapernotMJFCS15}
oraz DeepFool\cite{DBLP:journals/corr/Moosavi-Dezfooli15}.
Po zakończeniu implementacji i wykazaniu ich skutecznośći chciałbym
przetestować ich działanie dla bardziej skomplikowanych modeli klasyfikacyjnych
niż te z których korzystałem dotychczas(patrz punkt ~\ref{sec:fgsm}).
Końcowym zadaniem będzie utworzenie prostego interfejsu do obsługi aplikacji
(linia komend, bądź interfejs graficzny) oraz przetestowanie ataków na już
wytrenowanych modelach dostępnych publicznie


\section{Ataki z użyciem Fast Gradient Sign Method}
\label{sec:fgsm}
W celu sprawdzenia poprawności działania mojej implementacji metody FGSM utworzyłem
i wytrenowałem dwa proste modele klasyfikacyjne będące
splotowymi sieciami neuronowymi dla zbiorów CIFAR-10\cite{cifar_10} oraz MNIST\cite{mnist}
mające praktycznie taką samą strukturę oprócz wymiarów.
Struktura operacji ewaluacji uzyskana za pomocą narzędzia tensorboard zarówno dla
zbioru MNIST i CIFAR-10 znajduje się poniżej

\newpage
\vfill

\begin{figure}[ht]
  \begin{subfigure}{.55\textwidth}
    \includegraphics[width=\textwidth]{mnist_structure}
    \caption{MNIST}
    \centering
  \end{subfigure}
  \begin{subfigure}{.55\textwidth}
    \includegraphics[width=\textwidth]{cifar_structure}
    \caption{CIFAR}
    \centering
  \end{subfigure}
\end{figure}

\vfill
\clearpage

Poniżej można zobaczyc porównanie prawdopodobieństw przydziału do klasy dla
orignalnych danych wejściowych oraz dla spreparownych złośliwych danych.
Dla zbioru MNIST znacznie większy wpływ na skuteczność metody ma poziom
perturbacji niż liczba dokonanych iteracji. Zadowlajace wyniki pojawiają
się dopiero dla perturbacji na poziomie ~40 procent.
W przypadku zbioru CIFAR-10 można aby uzyskać przypisanie spreparowanych danych
do zadanej przez nas klasy z wysokim prawdopodobieństwem wystarczy 200 iteracji
przy poziomie perturbacji ~10 procent.

% \newpage
% \vfill
\begin{figure}[ht]
  \begin{subfigure}{.55\textwidth}
    \includegraphics[width=\textwidth]{mnist_orig}
    \caption{MNIST Klasyfikacja oryginalnego obrazu}
    \centering
  \end{subfigure}
  \begin{subfigure}{.55\textwidth}
    \includegraphics[width=\textwidth]{mnist_adver}
    \caption{MNIST Klasyfikacja spreparowanego obrazu dla zadanej klasy 1, 200 iteracji}
    \centering
  \end{subfigure}
  \begin{subfigure}{.55\textwidth}
    \includegraphics[width=\textwidth]{mnist_pert}
    \caption{MNIST wprowadzona perturbacja}
    \centering
  \end{subfigure}
\end{figure}
\begin{figure}[ht]
  \begin{subfigure}{.55\textwidth}
    \includegraphics[width=\textwidth]{cifar_orig.png}
    \caption{CIFAR Klasyfikacja oryginalnego obrazu}
    \centering
  \end{subfigure}
  \begin{subfigure}{.55\textwidth}
    \includegraphics[width=\textwidth]{cifar_adver}
    \caption{CIFAR Klasyfikacja spreparowanego obrazu dla zadanej klasy 1, 200 iteracji}
    \centering
  \end{subfigure}
  \begin{subfigure}{.55\textwidth}
    \includegraphics[width=\textwidth]{cifar_pert}
    \caption{CIFAR wprowadzona perturbacja}
    \centering
  \end{subfigure}
\end{figure}

% \vfill
\clearpage





\bibliography{pdi}
\bibliographystyle{ieeetr}

\end{document}
